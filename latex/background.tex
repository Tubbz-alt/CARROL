\section{Background}

Super-resolution aims at generating a high-resolution (HR) image from a
low-resolution (LR) image. Many different approaches have been tried out to
carry out this task. Some of them are - minimizing mean squared error(MSE),
Super Resolution Convolutional Neural Network (SRCNN) and Generative Adversarial
Network (GAN). A Supervised Learning Algorithm is used to minimize the cost
function which is the MSE between the generated image and ground truth. This
itself is a convenient way to maximize the peak signal to noise ratio (PSNR).
As shown by Ledig et al.\cite{Ledig}, this is not the most accurate way of comparison.
There is a limitation to MSE capturing image features and perceptual relevant
details as they are based on pixel-wise image difference. Interpolation is one
other simpler technique that is used for super resolution\cite{Siu}. An image is
divided into multiple mathematical subspaces. Within these subspaces,
interpolation is performed to fill in the missing pixel data and the final image
is reconstructed from these subspaces. These approaches lack to account for a
fundamental property of images. Pixels on a standalone basis don’t represent
much. On the other hand, they represent features of the objects in the image
when viewed with their neighbors. These approach lacks the acknowledgment of
these features and does not produce photo-realistic images\cite{Ferwerda}.

A better approach then this is to use a modified version of Convolutional Neural
Networks(CNN). CNN is the state of the art methods for image detection and
classification\cite{Krizhevsky, Jaswal2014}. CNNs comprise more than one convolutional layers followed
by one or more fully connected layers to classify images. The architecture is
designed to take advantage of the 2D structure of the image and the fact that
images can be classified by features which are represented by the neighboring
group of pixels. CNN consists of multiple convolutional layers with pooling
layers separating them. One convolutional layer is formed from the previous
layer by running a patch on the complete image and taking the dot product
between the patch values and the image values covered. Each convolution layer
consists of filters. The number of filters grows as the network grows. As the
data is processed through the network, the convolutional layers get reduced in
length and width but their depth increases. First few layers are able to
identify low-level features like lines. Moving forward, mid-level features like
arcs, curves, etc. are recognized. Finally, deep layers are able to recognize
complex features. The approach of Super Resolution Convolutional Neural
Network(SRCNN)\cite{Long} considers a convolutional neural network with an end-to-end
mapping between low- and high-resolution images. It consists of three main
steps: patch extraction and representation, non-linear mapping, and image
reconstruction\cite{Long}.

Another widely accepted approach is to use Generative Adversarial Network for
super-resolution\cite{Ledig}. For Super Resolution Generative Adversarial Network(SRGAN),
the goal is to train a generating function G that estimates HR for the given LR
image. The generative function is trained with a discriminator D that is trained
to distinguish super-resolved images from real images. The goal is to train them
so that D is no longer able to distinguish between the generated
super-resolution images and the ground truth. Eventually, the generator learns
to create images that are similar to the ground truth.

The problem with the above-mentioned methods is that they look at the image as a
whole. The intuition behind our idea is that each object in the image has its
own unique features and those features should be maintained when we run
super-resolution algorithms on them. We take sample images and run them through
a Fully Convolutional Network(FCN). FCNs are the state of the art technique used
for image segmentation. The output of the FCN is the segmented image. Using this
output, images for each of the objects belonging to different classes are
formed. These images are then given input to a Conditional GAN(CGAN). CGAN are a
form of GAN which creates images based on the context provided. For example, if
the input to this CGAN is ‘9’, it will produce images that it thinks resembles
‘9’. Each super-resolution image is then stitched back together to produce the
complete super-resolution image. This process is explained in detail below in
the methods.
